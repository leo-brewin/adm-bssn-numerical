\def\Date{2 Jun 2024}

\documentclass[12pt]{cdblatex}
\usepackage{pgfplots}
\usepackage{adm-bssn-eqtns}

\begin{document}

% =======================================================================================
\section*{ADM and BSSN variables for the Kasner metric}

A standard form of the Kasner metric is given by
\begin{equation*}
   ds^2 = - dt^2 + t^{2p_1} dx^2 + t^{2p_2} dy^2 + t^{2p_3} dz^2
\end{equation*}
where $p_1$, $p_2$ and $p_3$ are constants subject to
\begin{align*}
   1 &= p_1 + p_2 + p_3\\
   1 &= p^1_1 + p^2_2 + p^3_2
\end{align*}

The following Cadabra codes compute various quantities defined in the ADM and BSSN
formulations of the Einstein equations.

All of the results are exactly as expected (what else could it give?).

None of this is new -- the main point of this whole exercise is to use a familar metric
to explore how standard computations can be implemented using Cadabra.

None of these results are used by the main evolution codes (in the directories {\tt
adm} and {\tt bssn}) other than to set the initial data (at $t=1$).

The code that sets the initial data was written by hand (as opposed to the Cadabra
codes that generates the Ada procedures used in the evolution codes).

\clearpage

\begin{cadabra}
   {t,x,y,z}::Coordinate.
   {a,b,c,d,e,f,i,j,k,l,m,n,o,p,q,r,s,u#}::Indices(position=independent,values={t,x,y,z}).

   \partial{#}::PartialDerivative;

   {p1,p2,p3}::Symbol.

   p1::LaTeXForm("p_1").
   p2::LaTeXForm("p_2").
   p3::LaTeXForm("p_3").

   gBar{#}::LaTeXForm("{\bar g}").
   ABar{#}::LaTeXForm("{\bar A}").
   Aab{#}::LaTeXForm("{A}").
   phi::LaTeXForm("{\phi}").

   g_{a b}::Metric.
   g^{a b}::InverseMetric.

   g_{a b}::Depends(\partial{#}).

   # -----------------------------------------------------------------
   # rules used when evaluating components

   DtRule := {D^{t} -> 1}.      # components of d/dt, zero shift & unit lapse

   gabRule := { g_{t t} = gtt,
                g_{x x} = gxx,
                g_{y y} = gyy,
                g_{z z} = gzz }.

   # -----------------------------------------------------------------
   # the Kasner metric

   gab := { gtt -> -1,
            gxx -> t**(2*p1),
            gyy -> t**(2*p2),
            gzz -> t**(2*p3),
            gxy -> 0,
            gxz -> 0,
            gyz -> 0,
            gtx -> 0,
            gty -> 0,
            gtz -> 0}.

   # -----------------------------------------------------------------
   # standard definitions

   Detg := g ->  gxx gyy gzz - gxx gyz gyz
               - gxy gxy gzz + gxy gxz gyz
               + gxz gxy gyz - gxz gxz gyy.

   Gamma := \Gamma^{a}_{b c} ->
            (1/2) g^{a e} (   \partial_{b}{g_{e c}}
                            + \partial_{c}{g_{b e}}
                            - \partial_{e}{g_{b c}}).

   Rabcd := R^{a}_{b c d} ->
              \partial_{c}{\Gamma^{a}_{b d}} + \Gamma^{a}_{e c} \Gamma^{e}_{b d}
            - \partial_{d}{\Gamma^{a}_{b c}} - \Gamma^{a}_{e d} \Gamma^{e}_{b c}.


   Rab := R_{a b} -> R^{c}_{a c b}.

   Kab := K_{a b} -> - (1/2) D^{c} \partial_{c}{g_{a b}} / N.

   # -----------------------------------------------------------------
   # the BSSN variables

   trK   := K          -> g^{a b} K_{a b}.
   Aab   := Aab_{a b}  -> K_{a b} - (1/3) g_{a b} K.
   gBar  := gBar_{a b} -> g_{a b} / (g**(1/3)).
   ABar  := ABar_{a b} -> (K_{a b} - (1/3) g_{a b} K) / (g**(1/3)).
   phi   := phi        -> (1/12) \log(g).

   # -----------------------------------------------------------------
   # basic objects

   substitute (gabRule, gab)
   substitute (Detg,    gab)

   complete   (gabRule, $g^{a b}$)                                      # cdb(gabRule,gabRule)

   substitute (Rabcd,   Gamma)
   substitute (Rab,     Rabcd)

   # -----------------------------------------------------------------
   # convert to BSSN

   substitute (gBar,  Detg)                                             # cdb (gBar.01,gBar)

   substitute (Aab,   trK)                                              # cdb (Aab.01,Aab)
   substitute (Aab,   Kab)                                              # cdb (Aab.02,Aab)

   substitute (ABar,  trK)                                              # cdb (ABar.01,ABar)
   substitute (ABar,  Kab)                                              # cdb (ABar.02,ABar)
   substitute (ABar,  Detg)                                             # cdb (ABar.03,ABar)

   substitute (phi,   Detg)                                             # cdb (phi.01,phi)

   # -----------------------------------------------------------------
   # now evaluate the components

   evaluate   (gab,   join (gabRule,DtRule), rhsonly=True)              # cdb (gab,gab)
   evaluate   (Gamma, join (gabRule,DtRule), rhsonly=True)              # cdb (Gamma,Gamma)
   evaluate   (Rabcd, join (gabRule,DtRule), rhsonly=True)              # cdb (Rabcd,Rabcd)
   evaluate   (Rab,   join (gabRule,DtRule), rhsonly=True)              # cdb (Rab,Rab)
   evaluate   (Kab,   join (gabRule,DtRule), rhsonly=True)              # cdb (Kab,Kab)
   evaluate   (trK,   join (gabRule,DtRule), rhsonly=True)              # cdb (trK,trK)

   evaluate   (gBar,  join (gabRule,DtRule), rhsonly=True)              # cdb (gBar.02,gBar)
   evaluate   (Aab,   join (gabRule,DtRule), rhsonly=True)              # cdb (Aab.03,Aab)
   evaluate   (ABar,  join (gabRule,DtRule), rhsonly=True)              # cdb (ABar.04,ABar)

   evaluate   (phi,   join (gabRule,DtRule), rhsonly=True)              # cdb (phi.02,phi)

\end{cadabra}

\begin{dgroup*}
   \begin{dmath*} \Cdb*{gab}   \end{dmath*}
   \begin{dmath*} \Cdb*{Gamma} \end{dmath*}
   \begin{dmath*} \Cdb*{Rabcd} \end{dmath*}
   \begin{dmath*} \Cdb*{Rab}   \end{dmath*}
   \begin{dmath*} \Cdb*{Kab}   \end{dmath*}
   \begin{dmath*} \Cdb*{trK}   \end{dmath*}
\end{dgroup*}

% \clearpage

\begin{dgroup*}
   \begin{dmath*} \Cdb*{gBar.01} \end{dmath*}
   \begin{dmath*} \Cdb*{gBar.02} \end{dmath*}
\end{dgroup*}

\begin{dgroup*}
   \begin{dmath*} \Cdb*{phi.01} \end{dmath*}
   \begin{dmath*} \Cdb*{phi.02} \end{dmath*}
\end{dgroup*}

% \clearpage

\begin{dgroup*}
   \begin{dmath*} \Cdb*{Aab.01}  \end{dmath*}
   \begin{dmath*} \Cdb*{Aab.02}  \end{dmath*}
   \begin{dmath*} \Cdb*{Aab.03}  \end{dmath*}
\end{dgroup*}

\begin{dgroup*}
   \begin{dmath*} \Cdb*{ABar.01} \end{dmath*}
   \begin{dmath*} \Cdb*{ABar.02} \end{dmath*}
   \begin{dmath*} \Cdb*{ABar.03} \end{dmath*}
   \begin{dmath*} \Cdb*{ABar.04} \end{dmath*}
\end{dgroup*}

\end{document}
